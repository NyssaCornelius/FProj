\documentclass[reqno,11pt]{amsart}
\usepackage{amsmath, amsfonts, amsthm, amssymb, setspace, textcomp, enumerate, bbm, multirow, 
parskip, graphicx, pdfpages}
\usepackage{geometry}
%\usepackage[showframe]{geometry}
\geometry{hmargin={1in},vmargin={1in}}

\usepackage[hyphens]{url}
\usepackage{hyperref}
\hypersetup{breaklinks=true}

\pagestyle{plain}



\allowdisplaybreaks
\title{An analysis of US work stoppages and employee wages}

\author[N. Cornelius]{Nyssa Cornelius}
\email{nyssa.cornelius@du.edu}

\author[C. Jennings-Shaffer]{Chris  Jennings-Shaffer}
\email{christopher.jennings-shaffer@du.edu}


\begin{document}
\allowdisplaybreaks
\maketitle



\section{Introduction}

The GitHub repository for this project is located at
\url{https://github.com/NyssaCornelius/FProj}. The Binder link for this project
is \href{https://mybinder.org/v2/gh/NyssaCornelius/FProj/1d4a1dfcafdd4295908e81187a725c6250760315}{Binder link}.



The data sets are taken from the U.S. Bureau of Labor statistics (\url{https://www.bls.gov/}). 
For work stoppages, we used the excel file at \url{https://www.bls.gov/web/wkstp/monthly-listing.xlsx}.
This file contains information on each work stoppage in the US from 1993 on that involved at
least 1000 workers. Information about the columns is given in the following table
(all columns are stored as text).

\begin{tabular}{ccl}
column & basic type & description
\\\hline
Organizations involved 
	& 
	nominal
	&
	$\substack{\displaystyle\text{The company or branch of government where the}
	\\\displaystyle\text{work stoppage occurred.}}$
\\
States
	&
	nominal
	&
	The states where the work stoppage occurred.
\\
Areas
	&
	nominal
	&
	$\substack{\displaystyle\text{The geographic areas where the work stoppage }
	\\\displaystyle\text{occurred.}}$
\\
Ownership
	&
	nominal
	&
	$\substack{\displaystyle\text{This is either private industry (private sector) or}
	\\\displaystyle\text{state/local government (public sector).}}$
\\
Industry code	
	&
	numerical
	&
	$\substack{\displaystyle\text{The 2017 NAICS code describing the industry}
	\\\displaystyle\text{associated to the work stoppage.}}$
\\
Union	
	&
	nominal
	&
	$\substack{\displaystyle\text{The name of the union associated with the work} 
	\\\displaystyle\text{stoppage.}}$
\\
Union acronym	
	&
	nominal
	&
	The acronym for the union.
\\
Union Local	
	&
	nominal
	&
	$\substack{\displaystyle\text{A number or word description of the local union}
	\\\displaystyle\text{associated with the work stoppage.}}$
\\
Bargaining unit	
	&
	nominal
	&
	$\substack{\displaystyle\text{A description of the bargaining unit for}
	\\\displaystyle\text{the work stoppage (usually empty).}}$
\\
Work stoppage beginning date	
	&
	ordinal
	&
	Start date of the work stoppage.	
\\
Work stoppage ending date	
	&
	ordinal
	&
	End date of the work stoppage.	
\\	
Number of workers
	&
	numerical
	&
	$\substack{\displaystyle\text{The number of workers involved in the work}
	\\\displaystyle\text{stoppage.}}$
\\
$\substack{\displaystyle\text{Days idle cumulative}\\\displaystyle\text{for this work stoppage}}$
	&
	numerical
	&
	The total duration of the work stoppage.
\\
Note
	&
	nominal
	&
	$\substack{\displaystyle\text{Any notes about the work stoppage (e.g., the}
	\\\displaystyle\text{union changed names or the number of workers}
	\\\displaystyle\text{involved changed during the work stoppage).}}$
\end{tabular}


Employment statistics at the national level are taken from 
\url{https://download.bls.gov/pub/time.series/ce/},
specifically from 
\href{https://download.bls.gov/pub/time.series/ce/ce.industry}{ce.industry},
\href{https://download.bls.gov/pub/time.series/ce/ce.series}{ce.series},
and
\href{https://download.bls.gov/pub/time.series/ce/ce.data.0.AllCESSeries}{ce.data.0.AllCESSeries}
Since a full description of these data sets are given in 
\href{https://download.bls.gov/pub/time.series/ce/ce.txt}{ce.txt}, 
let us be a bit brief in the description. 
The file ce.series contains an entry for each type of data stored
at the national level (for example:
``Average hourly earnings of all employees, stone mining and quarrying, seasonally adjusted''),
whereas the actual data values for this are stored in 
ce.series; they are joined on the column series\_id (e.g., CES1021231003).
The file ce.series also contains a column for industry\_code, which is matched
to the description of the industry code in ce.industry. In ce.industry, there is
partial information to match an industry code of ce.series with an NAICS industry code of
work stoppage data. This is discussed in greater detail in Section 3.


Similarly, employment statistics at the state level are taken from 
\url{https://download.bls.gov/pub/time.series/sa/},
specifically
\href{https://download.bls.gov/pub/time.series/sa/sa.series}{sa.series},
\href{https://download.bls.gov/pub/time.series/sa/sa.data.0.Current}{sa.data.0.Current},
\href{https://download.bls.gov/pub/time.series/sa/sa.industry}{sa.industry},
and
\href{https://download.bls.gov/pub/time.series/sa/sa.state}{sa.state},
with a detailed description of the data sets given in
\href{https://download.bls.gov/pub/time.series/sa/sa.txt}{sa.txt}.
The file sa.series contains an entry for each type of data stored
at the state level and the actual data for the entries are stored in 
sa.data.0.Current, again joining on the column series\_id.
We use sa.state to decode the state for a series from numeric to string.
However, this time sa.industry does contain even partial information to 
match industry codes, but it does at least give an English description
of the codes. This is also discussed in greater detail in Section 3.


[...still need to add explanation for min wage data used in chloropleths ...]


The choice of using these data sets is that they contained the greatest
amount of information we could locate. Upon reviewing the literature,
they are the standard sets to use (and have been for a long time).
To load in the data, we downloaded the text files and then imported them
into Pandas dataframes in our jupyter notebook. The question we ask and
answer in this project is the following. With the major economic and political
changes of the last few decades, do strikes still have a negative correlation
with income inequality? Specifically, do strikes correlate with an increase
in wages of the associated workers? How do geographical regions affect
the frequency of strikes? Does the minimum wage play a role in the frequency
of strikes? The inputs for this project are the
data sets described above, the outputs are the data visualizations and 
conclusions that we draw below.


The remainder of this report is as follows. In section 2 we briefly discuss 
some of the existing literature (both research and non-research articles) on 
the subject, which will put our project in perspective of the larger picture.
In section 3 we go over the data cleaning necessary for our project and the 
challenges that this presented. In section 4 we both describe how we visualized
the data as well as give some of the most relevant visualizations. We conclude
this report in section 5, where we make our concluding remarks.


\section{Literature Review}

Articles discussing work stoppages are almost exclusively about strikes 
(as shut outs initiated by management are very rare in comparison) and 
usually focus on union representation of the work force. While this last part is usually true, it
is not universally so, as seen in \cite{Rubin} where the author argues that 
strikes and unions do not go together in terms of correlation with employee
wages and wealth inequality. In particular, they make the case that 
union representation decreases income inequality among specific groups but 
increases overall inequality, whereas strikes decrease overall income inequality 
(e.g. union representation may reduce income inequality among low and middle 
income white families while increasing income inequality between black and white
families, whereas strikes reduce income inequality at the aggregate level). 
That being said, this article is an outlier in this respect and considers
data only from 1949 to 1976. In the rest of our analysis, we take the more 
common approach of grouping unions and strikes together.

We will not spend much time on non-research literature, as for this subject it 
tends to be very clearly biased. However, we do note that while opinions
that are pro-union and pro-strike do sometimes cite relevant statistics,
this is far less the case with the opposing side. The latter is largely relegated 
to opinion pieces in local newspapers and blogs (and so we do not include 
references to them). While the pro-labor side also appears in similar sources
(which we also omit), non-research literature also comes out of think tanks such as
Economic Policy Institute and Washington Center for Equitable Growth.
These articles are more likely to back up their claims with data
and references (although they are often self-citing).
For example, in \cite{Misc1a} the author makes the case that strikes 
have and still do empower workers and reduce economic inequality. The article
discusses some of the political history surrounding unions and strikes 
in US, such when unions suffered a serious blow in 1981 when then President
Ronald Reagan fired 11,000 air traffic controllers for striking for higher pay
and reduced hours; describes summaries of polls on what workers do and do not
like about unions and strikes \cite{Misc1d}; and gives specific examples
of recent strikes that have and have not paid off for workers \cite{Misc1e}.
While not statistical data, the article also describes
some of the current anti-union practices of major corporations 
(see \cite{Misc1b}). Similarly, while the opinions in 
\cite{Misc3, Misc2} are not always backed by hard data, there the authors
do correctly point out trends in the number of strikes and issues
with the main source of data for work stoppages. The main source of
data for US work stoppages is the U.S. Bureau of Labor Statistics
(and this has been the case for over a century) and one issue
with the data is that it only includes work stoppages that involve at least 
1000 workers. As the authors point out, according to the Bureau of Labor
Statistics, nearly 60\% of workers in the private sector are employed
by companies with fewer than 1000 workers. Additional issues with this
data set are regularly brought up in research articles.

There are of course think tanks that are not pro-union, but their
arguments avoid statistics of economic inequality and the historical
correlation with unions and strikes. One example
of this is \cite{Misc4} from the Hoover Institution on War, Revolution, 
and Peace. Here the author argues against unions with pro-capitalism claims
of free trade and competition benefiting the worker. For statistics, 
the author instead looks to the overall performance of the US economy and unemployment levels
\cite{Misc4a, Misc4b, Misc4c}.
Another approach, as seen in \cite{Misc5} is to examine economic
inequality in terms of skilled labor and education of the workforce.
However, the authors of \cite{Misc1c} make the claim that the data and analysis in \cite{Misc1ci} 
shows that the levels of training and education do not adequately explain
include inequality. However, our project is not investigating statistics
about these other sources.

In terms of research articles, strikes and unions are studied in many
different respects. Lighter on the numbers are studies on public opinion
and ethical issues. In \cite{BurtonCrider} the authors give a lengthy account
of the opinions for and against the legality of strikes in the public sector strikes. 
Generally speaking, most people at the time of the article accepted that private sector 
employees should be able to strike, but were divided when it came to the the public sector.
Claims against public sector strikes are based on public sector work being essential, 
that the cost of increasing the collective bargaining power of public sector workers is higher
and with lower returns than in the private sector, and that public sector strikes are inappropriate 
because they may affect public policy. Also there is the dubious claim (which is disputed by the
authors \cite{BurtonCrider}) that low pay in the public sector, such as for teachers, is due to
public opinion on the importance of the service, whereas low pay in the private sector reflects 
a misallocation of resources. Ultimately the authors take the stance that strikes should be 
legal for areas of the public sector that would not lead to immediate public danger 
(e.g., fire fighters should not have the option to strike as part of their collective bargaining).
Related to the issue of public sector strikes, \cite{ThompsonSalmon} examines the ethical
issue of medical workers being able to strike, which is assumed to be an issue of major
importance as doctors move away from autonomous positions to the employee-employer
model common to modern health care. Another common topic about the public sector
is teacher strikes. By examining one specific case study, the authors of \cite{BelotWebbink}
draw the unsurprising conclusion that long term teacher strikes have long term (decades long) negative
affects on students, but enter the political realm by implying this should be used as
a talking point against the legality of teacher strikes. 
In \cite{HertelFernandezNaiduReich} the authors examine public opinion polls
about teacher strikes and education unions, and make the claim that public opinion
is generally pro-labor, with first hand knowledge of the strike greatly increasing
this (i.e., parents are more likely to side with their children's teachers than
to blame the teachers for the strike).


Taking statistics in mind, there are a large number of articles attempting
to study and model the occurrence of strikes \cite{Kennan, Kennan2, Mauro, Naples}, 
usually based on strikes appearing in waves over time. While present 
in articles over a century old \cite{Cross},
there has been an increased interest in
how relevant the recorded data actually is and how appropriately it is being
analyzed. For example,
in \cite{PalombaPalomba} it is pointed out that the data
from the Bureau of Labor statistics records the
the number of days of the work stoppage and the number of people involved,
but this does not properly measure the actual cost of the strike. The authors
attempt to correct this by examining the work force involved in each strike
and weighting the strike with a cost associated to the particular type of work.
Related to the issue of actual economic cost, the authors of \cite{Mchugh}
attempt to measure the impact of a strike based not just on the individual firm, 
but also on the cost to associated businesses.
The authors of \cite{SchorBowles} suggest that the the variance in strikes is better explained 
by the cost of losing a job than the unemployment rate, which is commonly used.
For the modern era, it is suggested in \cite{MartinDixon} that unions should be
studied separately for institutionalized unions and social movement unions,
whereas the authors of \cite{KeeranTarpinian} think that the old models
and explanations are no longer relevant to due to changes in business practices and public policy
stemming from technology and globalization. There is also the claim,
as seen in \cite{WallaceLeichtRaffalovich}, that strikes and unions are now irrelevant
to income inequality.


The point to take from this is that historically it has been widely accepted that
union representation and strikes correlate with improved conditions for the labor force. It is
also accepted that in recent decades, union representation and the number of strikes 
have significantly declined. This is attributed to the change in power held
by companies due to technology, globalization, and political power. In 2018 and 
2019 there was a sudden and major uptick in strikes, but like most things this
slowed in 2020 due to covid-19. Our research
is to verify that despite this loss in power of the labor force, strikes do indeed
still correlate with positive gains, as this should be verified and not be taken for granted.
[...To be changed, if our data doesn't show that...]



\section{Data Cleaning}
...

\section{Visualizations}
...

\section{Conclusions}
...





\bibliographystyle{alphadin}
\bibliography{reportRef}



\end{document}